\begin{entrylist}
    \projEntry
    {DJStreamr, collaborative streaming service for DJs}
    {\href{https://djstreamr.com}{djstreamr.com}}
    {AWS (Lambda, DynamoDB, EC2), Kotlin, VueJS, TS}
    {
        Music synchronisation protocol implemented as a webapp with an event-driven platform
        architecture which leverages serverless technology.
    }

    \projGhEntry{WACC, multiplatform compiler}
    {cottand/WACC}
    {Kotlin, JVM bytecode, ARM11 assembly}
    {
        Multiplatform compiler from WACC (a toy language) to JVM bytecode and ARM11 64-byte
    assembly that supports basic constructs like stack allocated primitives, and
    heap-allocated arrays and pairs.
    }

    \projEntry{PintOS, UNIX-like OS}
    {\href{https://pintos-os.org/}{pintos-os.org}}
    {C, x86 assembly}
    {
        Pint-sized OS with features such as advanced scheduling, paging, virtual memory and user
    programs with support for some of the C stdlib.
    }

    \projGhEntry{ICHack'19 \textit{(Hackathon)} -- Best Mobile App Award}
    {cottand/ICHack2019}
    {Unity, C\#}
    {
        AR-powered teaching app in that brings interactive 3D models to students’ phones.
    }

    \projEntry{IVANN, neural-network building GUI}
    {\href{https://icivann.github.io/ivann/}{IVANN website}}
    {Typescript, VueJS, Python}
    {
        Interactive code generator (for Python Tensorflow) to empower research by abstracting
    away from code, written in Typescript with VueJS
    }


    \projGhEntry{Multi-Paxos}
    {cottand/multi-paxos}
    {Elixir, Paxos}
    {
        Implementation of a variation of the Paxos consensus algorithm, as specified in the paper
    \href{https://dl.acm.org/citation.cfm?id=2673577}{Paxos Made Moderately Complex}
    }
\end{entrylist}

